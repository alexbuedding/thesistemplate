\documentclass[12pt,oneside]{book}                     % font size and type of document
\usepackage[a4paper,margin=25mm]{geometry}             % paper size and margins
\usepackage[onehalfspacing]{setspace}                   % line spacing

\usepackage{iftex}
\ifPDFTeX
   \usepackage[utf8]{inputenc}                          % used for mutated vowels e.g. ä, ü, ö, ß
   \usepackage[T1]{fontenc}                             % font encoding for pdflatex compiler
   \usepackage{mathptmx}                                % font type (Times)
   %\usepackage{lmodern}
\else
   \ifXeTeX
     \usepackage{fontspec}                              % font encoding for xelatex compiler
   \else 
     \usepackage{luatextra}                             % font encoding for lulatex compiler
   \fi
   \setmainfont{Times New Roman}                        % font type for xelatex or lulatex compiler
   \defaultfontfeatures{Ligatures=TeX}                  % enable TeX ligatures (character combinations such as "fi", "fl", "ff", etc.) replaced with special ligature glyphs for improved readability and aesthetics
   \usepackage{unicode-math}                            % math encoding and math fonts support
   \setmathfont{TeX Gyre Termes Math}                   % set the math font to TeX Gyre Termes Math, which matches Times New Roman
\fi
\usepackage{textcomp}                                   % additional symbols
\usepackage[ngerman]{babel}                             % load multilingual support for german
\usepackage[babel,german=guillemets]{csquotes}          % ensure multilingual support for biblatex quoting

\newcommand{\titlename}{thesis template}                % change title of document
\newcommand{\authorname}{prename surname}               % and name of author here
\title{\titlename}                                      % applied in whole document
\author{\authorname}
\date{\today}
\newcounter{savepage}                                   % save page number inside document

\usepackage[nostruts]{titlesec}
\titleformat{\chapter}[block]                                   % adjust chapter title format
    {\normalfont\huge\bfseries}{\thechapter}{20pt}{\Huge}       % default: [display]{\normalfont\huge\bfseries}{\chaptertitlename\ \thechapter}{20pt}{\Huge}
\titlespacing*{\chapter}{0pt}{-10pt}{20pt}                       % adjust chapter title spacing, default: {0pt}{50pt}{40pt}
%\titleformat{\section}                                          % adjust section title format
%    {\normalfont\Large\bfseries}{\thesection}{1em}{}            % default
%\titlespacing*{\section}{0pt}{22pt}{16pt}                       % adjust section title spacing, default
%\titleformat{\subsection}                                       % adjust subsection title format
%    {\normalfont\large\bfseries}{\thesubsection}{1em}{}         % default
%\titlespacing*{\subsection}{0pt}{19.5pt}{12pt}                  % adjust subsection title spacing, default
%\titleformat{\subsubsection}                                    % adjust subsubsection title format
%    {\normalfont\normalsize\bfseries}{\thesubsubsection}{1em}{} % default
%\titlespacing*{\subsubsection}{0pt}{19.5pt}{12pt}               % adjust subsubsection title spacing, default

\usepackage[titles]{tocloft}                            % extension for toc, lof, lot and other lists
\renewcommand{\cftchapleader}{\cftdotfill{\cftdotsep}}  % dotted chapter leaders
%\renewcommand{\cftsecleader}{\cftdotfill{\cftdotsep}}  % dot line for sections in toc
%\renewcommand{\cftchapfont}{\normalfont}                % toc entries in normal font style
%\renewcommand{\cftchappagefont}{\normalfont}            % toc page numbers in normal font style
%\cftsetindents{figure}{0pt}{40pt}                       % control space to left margin and the width of the numbers in lof
%\cftsetindents{table}{0pt}{40pt}                        % control space to left margin and the width of the numbers in lot
%\setlength{\cftchapnumwidth}{20pt}                      % adjust the width of the chapter numbers
%\setlength{\cftsecnumwidth}{30pt}                       % adjust the width of the section numbers
%\setlength{\cftsubsecnumwidth}{40pt}                    % adjust the width of the subsection numbers
%\setlength{\cftsubsubsecnumwidth}{40pt}                 % adjust the width of the subsubsection numbers
%\setlength{\cftfignumwidth}{40pt}                       % adjust the width of the figure numbers
%\setlength{\cfttabnumwidth}{40pt}                       % adjust the width of the table numbers
%\makeatletter
%\renewcommand{\cftmarktoc}{\@mkboth{\contentsname}{\contentsname}}
%\renewcommand{\cftmarklof}{\@mkboth{\listfigurename}{\listfigurename}}
%\renewcommand{\cftmarklot}{\@mkboth{\listtablename}{\listtablename}}
%\makeatother
\newcommand{\frontmatterchapter}[1]{
    \chapter*{#1}
    \markboth{#1}{#1}
    \addcontentsline{toc}{chapter}{#1}
}

\usepackage[                                            % header for bibliography
    backend=biber, 
    natbib=true,
    hyperref=true,
    style=apa,
    sorting=none
]{biblatex}
%\DeclareLanguageMapping{ngerman}{german-apa}           % only necessary for biblatex-apa style
%\DeclareFieldInputHandler{extradate}{\def\NewValue{}}  % remove suffix letters if author and year are identical in bibliography
\addbibresource{references.bib}                         % reference file
\makeatletter
\patchcmd{\blx@endenv@bibliography}
    {\endlist}{\endlist\clearpage}{}{}                  % insert a \clearpage command after end of bibliography list environment
\makeatother
\usepackage{tocbibind}                                  % adds all lists and bibliography to table of contents
  
\usepackage{fancyhdr}                                   % customize header and footer
\makeatletter
\renewcommand{\chaptermark}[1]{                         
    \markboth{#1}{}
}
% \renewcommand{\sectionmark}[1]{\markright{\thesection.\ #1}}
\fancypagestyle{plain}{                                 % redefine page style plain
\setlength{\headheight}{15pt}                           % height of header for more space
\fancyhf{}                                              % delete all header and footer content from page style
\fancyhead[R]{\thepage}
\fancyhead[L]{\ifnum\value{chapter}>0                   % chapter number only if it's greater 0
   \chaptername\ \thechapter: \fi
   \nouppercase{\leftmark}}
}
\pagestyle{fancy}                                       % define new page style called fancy
\setlength{\headheight}{15pt}                           % height of header for more space
\fancyhf{}                                              % delete all header and footer content from page style
\fancyhead[R]{\thepage}
\fancyhead[L]{\ifnum\value{chapter}>0                   % chapter number only if it's greater 0
   \chaptername\ \thechapter: \fi
   \nouppercase{\leftmark}
}
\newcommand{\backmatterpagestyle}{
\fancypagestyle{plain}{                                 % redefine page style plain for chapters after bibliography
\setlength{\headheight}{15pt}
\fancyhf{}
\fancyhead[R]{\thepage}
\fancyhead[L]{\nouppercase{\leftmark}}}
}
\makeatother

\usepackage{hyperref}
\hypersetup{
    colorlinks=true,
    linkcolor=black,
    citecolor=black,
    filecolor=magenta,      
    urlcolor=cyan,
    pdftitle={\titlename},
    pdfauthor={\authorname}
    }
\urlstyle{same}                                         % allows links within pdf document and to external websites
\usepackage{xurl}                                       % hyphening links

\usepackage{graphicx}                                   % required for inserting images
\graphicspath{{images/}}
\usepackage{caption}                                    % more customization of captions
%\captionsetup{figurename=Abb., tablename=Tab.}
\usepackage{wrapfig}                                    % wrap text around figures
\DeclareCaptionType{mycapequ}[][Formelverzeichnis]      % adds list of equations
\DeclareCaptionLabelFormat{nan}{Gleichung #2}           % caption compatibility with equations 
\captionsetup[mycapequ]{labelformat=nan}

\usepackage{xcolor}                                     % load xcolor package for customization
\usepackage{listings}                                   % supports listings environments
\definecolor{codegreen}{rgb}{0,0.6,0}                   % define and set custom colors
\definecolor{codegray}{rgb}{0.5,0.5,0.5}
\definecolor{codepurple}{rgb}{0.58,0,0.82}
\definecolor{backcolour}{rgb}{0.95,0.95,0.92}
\lstdefinestyle{mystyle}{                               % define listings style
    language={C++},
    morecomment=[l]{//},
    morekeywords={String},
    backgroundcolor=\color{backcolour},   
    commentstyle=\color{codegreen},
    keywordstyle=\color{magenta},
    numberstyle=\tiny\color{codegray},
    stringstyle=\color{codepurple},
    basicstyle=\ttfamily\footnotesize,
    breakatwhitespace=false,         
    breaklines=true,                 
    captionpos=b,                    
    keepspaces=true,                 
    numbers=left,                    
    numbersep=5pt,                  
    showspaces=false,                
    showstringspaces=false,
    showtabs=false,                  
    tabsize=2
}
\renewcommand{\lstlistlistingname}{Quellcodeverzeichnis}
\renewcommand{\lstlistingname}{Quellcode}                                   % change name of list of listings
\lstset{style=mystyle}                                                      % set listings style to my defined style

\usepackage[acronym,toc,translate=babel,nonumberlist,nopostdot]{glossaries} % adds support for glossary and acronyms and abbreviations
\renewcommand{\glsnamefont}[1]{\textbf{#1}}                                 % bold text for acronyms and glossary entries
\setlength{\glslistdottedwidth}{28pt}                                       % set indentation of list of acronyms/glossaries to left margin
\setlength\glsdescwidth{1.0\hsize}                                          % change width of list of acronyms or glossaries
\makenoidxglossaries
% add new acronyms here
\newacronym{utc}{UTC}{Coordinated Universal Time}
% add new glossary entries here
\newglossaryentry{latex}{name=LaTeX,description={Is a mark up language specially suited for scientific documents}}

\usepackage[ngerman, noabbrev]{cleveref}                % improves referencing of tables, figures, equations, etc.
%\crefformat{equation}{Gl.~(#2#1#3)}                    % customize reference to Gl. (default "Gleichung")
\crefname{listing}{Quellcode}{Quellcodes}               % change reference name
\Crefname{listing}{Quellcode}{Quellcodes}

\usepackage{lipsum}                                     % create dummy text in order to check formating
\usepackage[final]{pdfpages}                            % adds support for pdf file integration
\usepackage{amsmath}                                    % enhanced mathematical expressions
\usepackage{siunitx}                                    % includes si units support
\usepackage{longtable}                                  % for large tables that should continue on the next pages
\newcommand{\colwidth}[1]{\dimexpr#1\linewidth-2\tabcolsep-1.5\arrayrulewidth}  % custom command for controlling the column width
\usepackage{tabto}                                      % tabulator support
\usepackage{enumitem}                                   % for customizing itemize/enumerate
\newcommand{\multilineitemize}[1]{                      % custom command for controlling itemize environments inside tables
  \begin{minipage}[t]{\linewidth}
    \begin{itemize}[nosep,left=0pt,label=--,after=\strut]
      #1
    \end{itemize}
  \end{minipage}
}
\usepackage{tikz}                                       % create graphics in latex environment
\usetikzlibrary{positioning, calc}                      % enable relative positioning of nodes in tikz figure
\usepackage{pgfplots}                                   % create diagram, charts etc. with tikz library
\pgfplotsset{compat=1.18}                               % set used pgfplot package version for compatibility

\usepackage[activate={true,nocompatibility},            % activate protrusion and expansion
            final,                                      % enable microtype; use "draft" to disable
            tracking=true,                              % activate these techniques
            kerning=true,
            spacing=true,
            factor=1100,                                % add 10% protrusion amount (default is 1000)
            stretch=10,                                 % reduce stretchability/shrinkability 
            shrink=10]{microtype}                       % (default is 20/20)
\SetProtrusion{encoding={*},family={bch},series={*},size={6,7}}
              {1={ ,750},2={ ,500},3={ ,500},4={ ,500},5={ ,500},
               6={ ,500},7={ ,600},8={ ,500},9={ ,500},0={ ,500}}
\SetTracking{encoding={*}, shape=sc}{40}
\SetExtraKerning[unit=space]
    {encoding={*}, family={qhv}, series={b}, size={large,Large}}
    {1={-200,-200}, \textendash={400,400}}              % microtype package improves general appearence

\usepackage{afterpage}                                  % commands get expanded after the current page
\newcommand\blankpage{                                  % is output using \afterpage{\command}
    \newpage                                            % -> useful for \clearpage
    \null                                               % define command to insert blank page
    \thispagestyle{empty}
    \newpage}