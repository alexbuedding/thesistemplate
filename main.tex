%---- general settings ----%

\documentclass[12pt, oneside]{book}                     % font size and type of document
\usepackage[a4paper, margin = 25mm]{geometry}           % paper size and margins
\usepackage[onehalfspacing]{setspace}                   % line spacing

\usepackage{titlesec}
\titleformat{\chapter}[block]
    {\normalfont\huge\bfseries}{\thechapter}{20pt}{\Huge}
\titlespacing*{\chapter}{0pt}{*-4}{40pt}
  
\usepackage{fancyhdr}
\makeatletter
\renewcommand{\chaptermark}[1]{
\markboth{#1}{}}
% \renewcommand{\sectionmark}[1]{\markright{\thesection.\ #1}}
\fancypagestyle{plain}{
\setlength{\headheight}{15pt}
\fancyhf{}
\fancyhead[R]{\thepage}
\fancyhead[L]{\ifnum\value{chapter}>0                   % The chapter number only if it's greater 0
   \chaptername\ \thechapter: \fi
   \leftmark}
}
\pagestyle{fancy}
\setlength{\headheight}{15pt}
\fancyhf{}
\fancyhead[R]{\thepage}
\fancyhead[L]{\ifnum\value{chapter}>0                   % The chapter number only if it's greater 0
   \chaptername\ \thechapter: \fi
   \leftmark}
\makeatother

\usepackage{lipsum}                                     % create dummy text in order to check
                                                        % formating

\usepackage[T1]{fontenc}                                % hyphenating of words containing accented
                                                        % chars, font encoding for accented chars
\usepackage{textcomp}                                   % additional symbols
\usepackage{mathptmx}                                   % font type (Times)

\usepackage[utf8]{inputenc}                             % used for mutated vowels e.g. ä, ü, ö, ß
\usepackage[ngerman]{babel}                             % load multilingual support for german
\usepackage[babel,german=guillemets]{csquotes}          % ensure multilingual support for biblatex
                                                        % quoting

\newcommand{\titlename}{thesis template}                % change title of document
\newcommand{\authorname}{prename surname}               % and name of author here
                                                        % applied in whole document
\title{\titlename}
\author{\authorname}
\date{\today}
\newcounter{savepage}

\usepackage[hyphens]{url}                               % hyphening links
\usepackage{hyperref}
\hypersetup{
    colorlinks=true,
    linkcolor=black,
    citecolor=black,
    filecolor=magenta,      
    urlcolor=cyan,
    pdftitle={\titlename},
    pdfauthor={\authorname},
    pdfpagemode=FullScreen,
    }
\urlstyle{same}                                         % allows links within pdf document and to
                                                        % external websites

\usepackage{xcolor}                                     % load xcolor package for customization
\usepackage{listings}                                   % supports listings environments
\definecolor{codegreen}{rgb}{0,0.6,0}                   % define and set custom colors
\definecolor{codegray}{rgb}{0.5,0.5,0.5}
\definecolor{codepurple}{rgb}{0.58,0,0.82}
\definecolor{backcolour}{rgb}{0.95,0.95,0.92}

\lstdefinestyle{mystyle}{                               % define listings style
    language={C++},
    morecomment=[l]{//},
    morekeywords={String},
    backgroundcolor=\color{backcolour},   
    commentstyle=\color{codegreen},
    keywordstyle=\color{magenta},
    numberstyle=\tiny\color{codegray},
    stringstyle=\color{codepurple},
    basicstyle=\ttfamily\footnotesize,
    breakatwhitespace=false,         
    breaklines=true,                 
    captionpos=b,                    
    keepspaces=true,                 
    numbers=left,                    
    numbersep=5pt,                  
    showspaces=false,                
    showstringspaces=false,
    showtabs=false,                  
    tabsize=2
}

\renewcommand{\lstlistlistingname}{Quellcodeverzeichnis}
\renewcommand{\lstlistingname}{Quellcode}               % change name of list of listings
\lstset{style=mystyle}                                  % set listings style to my defined style

\usepackage{graphicx}                                   % required for inserting images
\graphicspath{{images/}}
\usepackage{caption}                                    % more customization of captions
%\captionsetup{figurename=Abb., tablename=Tab.}
\usepackage{wrapfig}                                    % wrap text around figures
\DeclareCaptionType{mycapequ}[][Formelverzeichnis]      % adds list of equations
\DeclareCaptionLabelFormat{nan}{Gleichung #2}           % caption compatibility with equations 
\captionsetup[mycapequ]{labelformat=nan}

\usepackage[final]{pdfpages}                            % adds support for pdf file integration
\usepackage{tikz}                                       % create graphics in latex environment
\usetikzlibrary{positioning, calc}                      % enable relative positioning of nodes in tikz figure
\usepackage{pgfplots}                                   % create diagram, charts etc. with tikz library
\pgfplotsset{compat=1.18}                               % set used pgfplot package version for compatibility
\usepackage{amsmath}                                    % enhanced mathematical expressions
\usepackage{siunitx}                                    % includes si units support
\usepackage[ngerman, noabbrev]{cleveref}                % improves referencing of tables,                                                                      % figures, equations, etc.
%\crefformat{equation}{Gl.~(#2#1#3)}                    % customise reference to Gl. (default
                                                        %  "Gleichung")
\crefname{listing}{Quellcode}{Quellcodes}               % change reference name
\Crefname{listing}{Quellcode}{Quellcodes}

\usepackage{tocloft}                            % extension for toc, lof, lot and other lists
\renewcommand{\cftchapleader}{\cftdotfill{\cftdotsep}}  % dotted chapter leaders
%\renewcommand{\cftsecleader}{\cftdotfill{\cftdotsep}}  % dot line for sections in toc
\setlength{\cftbeforetoctitleskip}{0pt}
\setlength{\cftbeforeloftitleskip}{0pt}
\setlength{\cftbeforelottitleskip}{0pt}
\makeatletter
\renewcommand{\cftmarktoc}{\@mkboth{\contentsname}{\contentsname}}
\renewcommand{\cftmarklof}{\@mkboth{\listfigurename}{\listfigurename}}
\renewcommand{\cftmarklot}{\@mkboth{\listtablename}{\listtablename}}
\newcommand{\frontmatterchapter}[1]{
    \chapter*{#1}
    \markboth{#1}{#1}
    \addcontentsline{toc}{chapter}{#1}
}
\makeatother

\usepackage[acronym,toc,translate=babel,
            nonumberlist, nopostdot]{glossaries}
\renewcommand{\glsnamefont}[1]{\textbf{#1}}
\setlength\LTleft{0pt}
\setlength\LTright{0pt}
\setlength\glsdescwidth{0.8\hsize}                      % adds support for glossary and
\makenoidxglossaries                                         % acronyms and abbreviations
% add new acronyms here
\newacronym{utc}{UTC}{Coordinated Universal Time}
% add new glossary entries here
\newglossaryentry{latex}{name=LaTeX,description={Is a mark up language specially suited for scientific documents}}

\usepackage[                                            % header for bibliography
    backend=biber, 
    natbib=true,
    hyperref=true,
    style=apa,
    sorting=none,
]{biblatex}
%\DeclareLanguageMapping{ngerman}{german-apa}           % only necessary for biblatex-apa style
%\DeclareFieldInputHandler{extradate}{\def\NewValue{}}  % remove suffix letters if author and year
\addbibresource{references.bib}                         % are identical in bibliography

\defbibheading{bibintocchap}[\bibname]{
    \chapter*{#1}
    \markboth{#1}{#1}
    \addcontentsline{toc}{chapter}{#1}
}
\newcommand{\printmybib}[1]{
\makeatletter
\fancypagestyle{plain}{
\setlength{\headheight}{15pt}
\fancyhf{}
\fancyhead[R]{\thepage}
\fancyhead[L]{\leftmark}
}
\makeatother
\printbibliography[heading = bibintocchap,
                   title = #1]
}
\makeatletter
\patchcmd{\blx@endenv@bibliography}{\endlist}{\endlist\clearpage}{}{}
\makeatother

\usepackage{tocbibind}                                  % adds all lists and bibliography to table of
                                                        % contents

\usepackage[activate={true,nocompatibility},            % activate protrusion and expansion
            final,                                      % enable microtype; use "draft" to disable
            tracking=true,                              % activate these techniques
            kerning=true,
            spacing=true,
            factor=1100,                                % add 10% protrusion amount (default is 
            stretch=10,                                 % 1000);reduce stretchability/shrinkability 
            shrink=10]{microtype}                       % (default is 20/20)
\SetProtrusion{encoding={*},family={bch},series={*},size={6,7}}
              {1={ ,750},2={ ,500},3={ ,500},4={ ,500},5={ ,500},
               6={ ,500},7={ ,600},8={ ,500},9={ ,500},0={ ,500}}
\SetTracking{encoding={*}, shape=sc}{40}
\SetExtraKerning[unit=space]
    {encoding={*}, family={qhv}, series={b}, size={large,Large}}
    {1={-200,-200}, \textendash={400,400}}              % microtype package improves general                                                            % appearence

\usepackage{afterpage}                                  % commands get expanded after the current                                                           % page  
\newcommand\blankpage{                                  % is output using \afterpage{\command}
    \newpage                                            % -> useful for \clearpage
    \null                                               % define command to insert blank page
    \thispagestyle{empty}
    \newpage}

%---- title page ----%

\begin{document}

\frontmatter

%\title{Praxisphasenbericht Template}
%\author{Alex Buedding}
%\date{March 2023}
%\maketitle

\begin{titlepage}

\centering
\includegraphics[width=0.4\textwidth]{w-hs.png}
\par\vspace{1.5cm}
{\scshape\Large Westfälische Hochschule - Fachbereich 5\par}
\vspace{1cm}
{\scshape\Large Modul\par}
{\scshape\normalsize weitere Informationen\par}
{\scshape\small Zeitraum\par}
\vspace{2cm}
{\Large\textbf{\titlename}\par}
\vspace{2cm}
{\Large\itshape \authorname~(Matrikelnummer)\par}
{\large\itshape Straße Hausnummer \par}
{\large\itshape PLZ Ort\par}
\vfill
{Eingereicht bei\par
Prof. Dr.-Ing. Michael Bühren\par}
\vfill
{\large Datum: \today\par}

\end{titlepage}

%---- blank page ----%

\blankpage

%---- restriction note ----%

\setcounter{page}{-1}
\thispagestyle{empty}

\noindent\textbf{\Huge Sperrvermerk}

\bigskip{\noindent Der vorliegende Praxisphasenbericht beinhaltet interne und vertrauliche Informationen des Unternehmens:}

\bigskip{\noindent\textbf{Lebbing automation\& drives GmbH\newline
Konrad-Zuse Straße 16\newline
46397 Bocholt}}

\bigskip{\noindent Eine Einsicht in diesen Praxisphasenbericht ist nicht gestattet. Ausgenommen davon sind die betreuenden Professoren, sowie die befugten Mitarbeiter des Prüfungsausschusses des Fachbereichs Wirtschaft und Informationstechnik. Die Weitergabe des Inhaltes der Arbeit und der enthaltenen Informationen und Daten im Gesamten oder Teilen ist grundsätzlich untersagt. Es dürfen keine Kopien oder Abschriften, auch nicht in digitaler Form, gefertigt werden. Ausnahmen von dieser Regelung bedürfen einer schriftlichen Genehmigung des Unternehmens Lebbing automation \& drives GmbH.}

\clearpage

%---- preamble ----%

\pagenumbering{Roman}

\frontmatterchapter{Vorwort}

\clearpage

%---- abstract ----%

%\begin{abstract}
%    Here comes the abstract.
%\end{abstract}

\frontmatterchapter{Abstract}

\clearpage

%---- toc, lof, lot, loa, loe, lol ----%

%---- table of contents ----%

\microtypesetup{protrusion=false} % disables protrusion locally in the document

\tableofcontents

\microtypesetup{protrusion=true} % enables protrusion

\clearpage

%---- list of figures ----%

\microtypesetup{protrusion=false} % disables protrusion locally in the document

\listoffigures

\microtypesetup{protrusion=true} % enables protrusion

\clearpage

%---- list of tables ----%

\microtypesetup{protrusion=false} % disables protrusion locally in the document

\listoftables

\microtypesetup{protrusion=true} % enables protrusion

\clearpage

%---- list of acronyms ----%

\printnoidxglossary[type=acronym,style=long,title={Abkürzungsverzeichnis}]

\clearpage

%---- list of equations ----%

\microtypesetup{protrusion=false} % disables protrusion locally in the document

\listofmycapequs

\microtypesetup{protrusion=true} % enables protrusion

\clearpage

%---- list of listings ----%

\microtypesetup{protrusion=false} % disables protrusion locally in the document

\lstlistoflistings

\microtypesetup{protrusion=true} % enables protrusion

\clearpage

%---- text ----%

\setcounter{savepage}{\arabic{page}}

\mainmatter

\chapter{Kapitel}\label{chap:first}
\lipsum[]\parencite{biblatex}
\acrshort{utc}
\gls{latex}
\begin{mycapequ}[!ht]
    \begin{equation}
        {P(\bigcup_{n=1}^n A_n) \leq \sum_{n=1}^n P(A_n)}
        \label{eq:bool} %\label used for referencing the equation in text
    \end{equation}
    \caption{Bool'sche Gleichung}
\end{mycapequ}

\noindent Die tolle Bool'sche \cref{eq:bool} macht uns alle glücklich.

\begin{mycapequ}[!ht]
    \begin{equation}
        \mathrm{E=m\cdot c^2}
        \label{eq:meequi} %\label used for referencing the equation in text
    \end{equation}
    \caption{Albert Einstein's mass-energy equivalence}
\end{mycapequ}

\section{Abschnitt}\label{sec:first}

\begin{figure}[!ht]
    \centering
    \includegraphics[width=0.4\linewidth]{w-hs.png}
    \captionsetup{width=1.0\linewidth}
    \caption[Logo der Westfälischen Hochschule]{Logo der Westfälischen Hochschule \parencite{whs}}
    \label{fig:w-hs} %\label used for referencing the figure in text
\end{figure}

\noindent Und hier haben wir ein tolles Logo \cref{fig:w-hs} und jetzt springen wir alle im Kreis.

\begin{table}[!ht]
\centering
    \begin{tabular}{ | c | c | c | }
        \hline
        symbol & value & unit \\ \hline            
        $z Na$ & 11 & - \\ \hline      
        $z F$ & 9 & - \\ \hline      
        $Emax Na$ & 0.545 & $[MeV]$ \\ \hline
    \end{tabular}
    \caption{Beispieltabelle}
    \label{tab:example} %\label used for referencing the figure in text
\end{table}

\noindent Abschließend gibt es in \cref{tab:example} noch etwas tolles zu sehen und das ist total dufte.

\subsection{Unterabschnitt}\label{subsec:first}

\begin{lstlisting}[caption = C++ Quellcodebeispiel, captionpos = b, label = lst:example]
// polls CAN shift registers and writes content into global string msg. then returns it.
String processCan(){                
  String msg = "";
  int id = 0;
  int packetSize = CAN.parsePacket();

  if (packetSize) {
    // save current message id
    id = CAN.packetId(); 
    // push all message bits into single string
    while (CAN.available()) {       
      msg += char(CAN.read());
    }
    // manage received can messages
    if(id>=100){                    
      healthHandler(id,msg);
    }else if(id<100){
      valueHandler(id,msg);
    } 
    return msg;
  }
}
\end{lstlisting}

\noindent Und noch ein Beispiel für einen \cref{lst:example}, den wir alle lieben.

\begin{figure}[!ht]
    \centering
    \includegraphics[scale=0.3]{to-do-list.png}
    \caption{todo}
    \label{fig:todo}
\end{figure}

\noindent Hier ein Beispiel eines \textcite{todo}-Platzhalters für Literaturnachweise, die noch hinzugefügt werden müssen.

\clearpage

\chapter{Anderes Kapitel}\label{chap:another}

\begin{figure}[!ht]
    \centering
    \includegraphics[width=1.0\textwidth]{example-image-a}
    \captionsetup{width=1.0\textwidth}
    \caption[skalierte Beispielabbildung]{skalierte Beispielabbildung (eigene Abbildung)}
    \label{fig:scaledexampleimagea}
\end{figure}

\begin{figure}[!ht]
\centering
    \begin{minipage}[c]{.475\textwidth}
    \centering
        \includegraphics[width=1.0\linewidth]{example-image-b}
        \captionof{figure}[Beispielabbildung b]{Beispielabbildung b (eigene Abbildung)}
        \label{fig:horizontalalignedimageb}
    \end{minipage}\hspace{.025\textwidth}
    \begin{minipage}[c]{.475\textwidth}
        \centering
        \includegraphics[width=1.0\linewidth]{example-image-c}
        \captionof{figure}[Beispielabbildung c]{Beispielabbildung c (eigene Abbildung)}
        \label{fig:horizontalalignedimagec}
    \end{minipage}
\end{figure}

\clearpage

\chapter{Letztes Kapitel}\label{chap:last}

\lipsum[1]

\begin{wrapfigure}{l}{0.4\textwidth}
    \centering
    \includegraphics[width=0.9\linewidth]{example-image}
    \captionsetup{width=0.9\linewidth}
    \caption[Beispielabbildung mit Text umrandet]{Beispielabbildung mit Text umrandet (eigene Abbildung)}
    \label{fig:textwrappedaroundexampleimage}
\end{wrapfigure}

\lipsum[2-4]

\clearpage

\section{Letzter Abschnitt}\label{sec:last}

\cref{fig:tikzexamplegraphics} zeigt eine Beispielgrafik, die mit dem Paket \glqq Tikz\grqq{} erstellt wurde.

\begin{figure}[!ht]
    \centering
	\begin{tikzpicture}
        \draw (0,0) circle (1);
        \draw (2,0) circle (1.5in);
        \draw (5,0) ellipse (10pt and 20 pt);
        \draw node at (3,0) {$f(x)$};
        \filldraw (6,0) circle (0.1cm) node[anchor=west]{Anchored Node};
    \end{tikzpicture}
    \caption{Tikz Beispielgrafik}
    \label{fig:tikzexamplegraphics}
\end{figure}

\noindent Als ein weiteres komplexeres Beispiel kann hier auch noch folgende \cref{fig:tikzexamplediagram} angeführt werden.

\begin{figure}[!ht]
    \centering
    \begin{tikzpicture}[
                        youngnode/.style={rectangle, draw=red!60, fill=red!5, very thick, minimum size=40},
                        oldnode/.style={rectangle, draw=blue!60, fill=blue!5, very thick, minimum size=40},
                        ]
        %Nodes
        \node[oldnode]        (SusO)                            { $S_O(t)$};
        \node[oldnode]        (InfO)       [below=of SusO]      { $I_O(t)$};
        \node[oldnode]        (RecO)       [below=of InfO]      { $R_O(t)$};

        \node[youngnode]      (SusY)        [left=of SusO]      { $S_Y(t)$};
        \node[youngnode]      (InfY)        [left=of InfO]      { $I_Y(t)$};
        \node[youngnode]      (RecY)        [left=of RecO]      { $R_Y(t)$};

        %Lines
        \draw[->, very thick] (SusO.south east)  to node[right] {$a_{OO}$} (InfO.north east);
        \draw[->, very thick] (InfO.south)  to node[right] {$b_O$} (RecO.north);
        \draw[->, very thick] (RecO.east)  .. controls  +(right:17mm) and +(right:17mm)   .. (SusO.east);

        \draw[->, very thick] (SusY.south west)  to node[left] {$a_{YY}$} (InfY.north west);
        \draw[->, very thick] (InfY.south)  to node[left] {$b_Y$} (RecY.north);
        \draw[->, very thick] (RecY.west) .. controls  +(left:17mm) and +(left:17mm)   .. (SusY.west);

        \draw[dashed,->, very thick] (InfO.north west)  to  (SusY.south east);
        \draw[->, very thick] (SusY.south east)  to node[left] {$a_{OY}$} (InfY.north east);

        \draw[->, very thick] (SusO.south west)  to node[right] {$a_{YO}$} (InfO.north west);
        \draw[dashed,->, very thick] (InfY.north east)  to  (SusO.south west);
    \end{tikzpicture}
    \caption{Tikz Beispieldiagramm}
    \label{fig:tikzexamplediagram}
\end{figure}

\clearpage

\noindent Das Paket \glqq pgfplots\grqq{} basiert auf \glqq Tikz\grqq{} und ermöglicht das Zeichnen von mathematischen Diagrammen aller Art. In \cref{fig:pgfplotsexample2D} ist dies anhand eines Beispiels dargestellt.

\begin{figure}[!ht]
    \centering
    \begin{tikzpicture}
        \begin{axis}[clip=false,xmin=0,xmax=2.5*pi,ymin=-1.5,ymax=1.5, axis lines=middle,xtick={0,pi/2,pi,3*pi/2,2*pi},xticklabels={$0$,$\frac{\pi}{2}$,$\pi$,$\frac{3}{2}\pi$,$2\pi$},xticklabel style={anchor=south west},xmajorgrids=true,grid style=dashed]
            \addplot[domain=0:2*pi,red]{sin(deg(x))}
            node[right,pos=0.9]{$f(x)=\sin x$};
            \addplot[domain=0:2*pi,blue]{cos(deg(x))}
            node[right,pos=1.0]{$g(x)=\cos x$};
        \end{axis}
    \end{tikzpicture}
    \caption{pgfplots Beispiel einer 2D-Grafik}
    \label{fig:pgfplotsexample2D}
\end{figure}

\noindent Außerdem können auch 3D-Grafiken damit erstellt werden, wie in \cref{fig:pgfplotsexample3D} und in \cref{fig:anotherpgfplotsexample3D} zu sehen ist.

\begin{figure}[!ht]
    \centering
    \begin{tikzpicture}
        \begin{axis}[colormap/cool]
            \addplot3[mesh,samples=20]{1-x^2-y^2};
        \end{axis}
    \end{tikzpicture}
    \caption{pgfplots Beispiel einer 3D-Grafik}
    \label{fig:pgfplotsexample3D}
\end{figure}

\clearpage

\begin{figure}[!ht]
    \centering
    \begin{tikzpicture}
        \begin{axis}[view={50}{30}]
            \addplot3+[domain=0:5*pi,samples=60,samples y=0]({sin(deg(x)},{cos(deg(x)},{x});
        \end{axis}
    \end{tikzpicture}
    \caption{weiteres pgfplots Beispiel einer 3D-Grafik}
    \label{fig:anotherpgfplotsexample3D}
\end{figure}

\noindent Darüber hinaus können auch Daten aus .txt-Dateien ausgelesen werden und eine Grafik überführt werden (s. \cref{fig:pgfplotsexampledata}).

\begin{figure}[!ht]
    \centering
    \begin{minipage}[t]{0.95\textwidth}
        \centering
        \begin{tikzpicture}
            \begin{axis}[xmin=0, xmax=32, xlabel=$k$, ylabel=$x_k$, ymin=0, ymax=1, ymajorgrids=true, xmajorgrids=true, width=0.99\linewidth]
                \addplot+[only marks] table[x=k, y=hk]{data/exampledata.txt};
            \end{axis}
        \end{tikzpicture}
        \caption{pgfplot Beispiel einer Datenreihe}
        \label{fig:pgfplotsexampledata}
    \end{minipage}
\end{figure}

\noindent Weitere Informationen zu den Paketen gibt es in der Dokumentation unter \parencite{tikz} und \parencite{pgfplots}.

\clearpage

% --- Bibliography --- %

\pagenumbering{Roman}
\setcounter{page}{\thesavepage}

\printmybib{Literaturverzeichnis}

\clearpage

%---- appendix ----%

\appendix

\input{appendix}

\clearpage

%---- glossary ----%

\backmatter

\printnoidxglossary[style=long, title={Glossar}]

\clearpage

%---- affidavit ----%

\thispagestyle{empty}

\noindent\textbf{\Huge Eidesstattliche Versicherung\newline}
\bigskip{\noindent Büdding, Alexander\newline}
\noindent\rule[1ex]{\textwidth}{1pt}\newline
Name, Vorname

\bigskip{\noindent Ich versichere hiermit an Eides statt, dass ich die vorliegende Abschlussarbeit mit dem Titel:}
\bigskip{\noindent\textbf{"hier Titel einfügen hier Titel einfügen hier Titel einfügen hier Titel einfügen"}}

\bigskip{\noindent selbstständig und ohne unzulässige fremde Hilfe erbracht habe. Ich habe keine anderen als die angegebenen Quellen und Hilfsmittel benutzt sowie wörtliche und sinngemäße Zitate kenntlich gemacht. Die Arbeit hat in gleicher oder ähnlicher Form noch keiner Prüfungsbehörde vorgelegen.}

\bigskip{\noindent Bocholt, XX.XX.2023\newline}
\noindent\rule[1ex]{\textwidth}{1pt}
Ort, Datum, Unterschrift

\clearpage

\end{document}
